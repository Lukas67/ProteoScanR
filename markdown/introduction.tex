% Options for packages loaded elsewhere
\PassOptionsToPackage{unicode}{hyperref}
\PassOptionsToPackage{hyphens}{url}
%
\documentclass[
]{article}
\usepackage{amsmath,amssymb}
\usepackage{lmodern}
\usepackage{iftex}
\ifPDFTeX
  \usepackage[T1]{fontenc}
  \usepackage[utf8]{inputenc}
  \usepackage{textcomp} % provide euro and other symbols
\else % if luatex or xetex
  \usepackage{unicode-math}
  \defaultfontfeatures{Scale=MatchLowercase}
  \defaultfontfeatures[\rmfamily]{Ligatures=TeX,Scale=1}
\fi
% Use upquote if available, for straight quotes in verbatim environments
\IfFileExists{upquote.sty}{\usepackage{upquote}}{}
\IfFileExists{microtype.sty}{% use microtype if available
  \usepackage[]{microtype}
  \UseMicrotypeSet[protrusion]{basicmath} % disable protrusion for tt fonts
}{}
\makeatletter
\@ifundefined{KOMAClassName}{% if non-KOMA class
  \IfFileExists{parskip.sty}{%
    \usepackage{parskip}
  }{% else
    \setlength{\parindent}{0pt}
    \setlength{\parskip}{6pt plus 2pt minus 1pt}}
}{% if KOMA class
  \KOMAoptions{parskip=half}}
\makeatother
\usepackage{xcolor}
\usepackage[margin=1in]{geometry}
\usepackage{graphicx}
\makeatletter
\def\maxwidth{\ifdim\Gin@nat@width>\linewidth\linewidth\else\Gin@nat@width\fi}
\def\maxheight{\ifdim\Gin@nat@height>\textheight\textheight\else\Gin@nat@height\fi}
\makeatother
% Scale images if necessary, so that they will not overflow the page
% margins by default, and it is still possible to overwrite the defaults
% using explicit options in \includegraphics[width, height, ...]{}
\setkeys{Gin}{width=\maxwidth,height=\maxheight,keepaspectratio}
% Set default figure placement to htbp
\makeatletter
\def\fps@figure{htbp}
\makeatother
\setlength{\emergencystretch}{3em} % prevent overfull lines
\providecommand{\tightlist}{%
  \setlength{\itemsep}{0pt}\setlength{\parskip}{0pt}}
\setcounter{secnumdepth}{-\maxdimen} % remove section numbering
\usepackage{float}
\usepackage{sectsty}
\ifLuaTeX
  \usepackage{selnolig}  % disable illegal ligatures
\fi
\IfFileExists{bookmark.sty}{\usepackage{bookmark}}{\usepackage{hyperref}}
\IfFileExists{xurl.sty}{\usepackage{xurl}}{} % add URL line breaks if available
\urlstyle{same} % disable monospaced font for URLs
\hypersetup{
  hidelinks,
  pdfcreator={LaTeX via pandoc}}

\author{}
\date{\vspace{-2.5em}}

\begin{document}

\hypertarget{introduction}{%
\section{Introduction}\label{introduction}}

\hypertarget{from-the-central-dogma-of-molecular-biology-to-omics-and-systems-biology}{%
\subsection{From the Central Dogma of Molecular Biology to Omics and
Systems
Biology}\label{from-the-central-dogma-of-molecular-biology-to-omics-and-systems-biology}}

Before understanding the underlying principle biological systems undergo
and the initial exploration of molecular biology, the processes of live
were roughly inferred and not well understood. The central dogma of
molecular biology serves as the cornerstone of biological processes,
providing a framework for understanding how genetic information is
converted into functional proteins. This framework consists of two major
processes, which involve transcription, converting DNA into RNA, and
translation, converting RNA into functional proteins. In between those
processes regulation takes base on a molecular level, by altering
structure with chemical adaptations. The regulation is undertaken to
provide biological organisms with an highly efficient output of their
genetic products called proteins. Crick's formulation of the central
dogma has guided research in molecular biology for over 60 years
\citep{Cobb2017}. Over the past six decades, researchers have explored
various aspects of this principle, collectively referred to as omics, to
comprehensively analyze biomolecules in diverse contexts. Omics research
has emerged as an approach for comprehensively studying biomolecules and
their interactions within biological systems. Omics encompasses various
fields, including genomics, transcriptomics, proteomics, metabolomics,
and others, each focusing on a specific class of biomolecules. By
analyzing multiple omics layers, researchers can gain a holistic view of
biological processes, uncover regulatory mechanisms, and identify key
players in complex biological networks with computational methods.
Systems biology takes a quantitative approach to investigate the
interactions, dynamics, and emergent properties of biomolecular
networks. Imagining biological systems as complex, multi-layer networks,
the ensemble of proteins, known as the proteome, plays a critical role
in this interactive structure with many functions such as structural
integrity, catalyzing chemical reaction and regulation of cellular
funcitons.

\hypertarget{proteomics}{%
\subsection{Proteomics}\label{proteomics}}

The central dogma of molecular biology offers insights not only at the
level of individual terms but also provides a hierarchical understanding
of biology. Hierarchical organization is the underlying principle in all
topics of biology. Proteins exhibit hierarchical structures that
contribute to their functionality. At the primary structure level,
proteins are composed of smaller building blocks called amino acids.
Combining them results in the protein secondary structure such as alpha
helix and beta sheet. These secondary structures further assemble to
create higher-order tertiary structures, which represent specific
domains within the protein. Finally, the quaternary structure describes
the functional state of the protein as a whole at a given time. By
studying the proteome, which encompasses all the proteins present in a
biological system at a specific moment, we can gain a comprehensive
understanding of the hierarchical organization and functional dynamics
of biological processes.

\hypertarget{bulk-proteomics}{%
\subsubsection{Bulk proteomics}\label{bulk-proteomics}}

Higher organisms are composed of specialized cells organized into
tissues, such as skin, muscle, and blood. Each tissue consists of cells
with specific functions, resulting in variations in protein expression.
Bulk proteomics is a technique used to analyze the protein composition
of a tissue sample, which contains all types of cells present in that
particular tissue. This approach finds valuable applications for
instance in oncology, where it can provide insights into the protein
profile of a tumor. By studying oncogenic biomarkers and expression
patterns, this knowledge can contribute to the development of screening
methods and the customization of treatments based on observed expression
patterns. Additionally, bulk proteomics can aid in identifying the
progression of tumor development \citep{Kwon2021}. Bulk analysis
expression profiles provide scientists with an average protein abundance
across all cells in a sample, offering a comprehensive view of protein
expression. Additionally, compared to single-cell methods, bulk samples
typically contain a greater number of proteins. In fact the expanded
library can be used to validate the performance of single-cell type and
single-cell proteomics methods and bioinformatics pipeline
\citep{Schoof2021}. However, in this type of analysis, the precise
properties of individual cell types can only be inferred and remain
obscured.

\hypertarget{single-cell-type}{%
\subsubsection{Single-cell type}\label{single-cell-type}}

Taking tissue samples can lead to an averaging effect across the entire
cellular ensemble, making it difficult to discern specific cell types.
To overcome this limitation, cell sorting techniques such as
fluorescence-activated cell sorting (FACS), magnetic-activated cell
sorting (MACS), and buoyancy-activated cell sorting (BACS) are employed
\citep{Liou2015}. These techniques rely on the detection of surface
proteins on cells and the specific binding of antibodies to these
proteins. Antibodies, typically derived from the immune system of other
species, bind to antigens in a lock-and-key manner. Antibodies are
widely used in molecular biology and find applications in cell
isolation. Depending on the desired application, the antibodies can be
labeled with fluorescent markers (FACS), magnetic beads (MACS), or
biotin (BACS). These methods have varying technical requirements, with
flow cytometers being the most complex and costly. In MACS, magnetic
beads are used to isolate cells from a solution by utilizing a magnet.
However, mechanical forces involved in MACS can potentially damage the
cells. BACS, a method developed in 2015, employs microbubbles to carry
the antibodies for cellular isolation. The method showed improvements
compared to MACS in regards of viability after sorting. It is important
to note that FACS is considered the gold standard for validating the
effectiveness of MACS and BACS due to its high purity and yield. FACS
was used in the validation of these two methods
\citep{Liou2015, Sutermaster2019}. Single-cell type proteomics reflects
the protein ensemble of a specific cell type at a particular time,
providing a focused perspective on the studied field compared to bulk
methods \citep{Maes2020}.

\hypertarget{single-cell-proteomics}{%
\subsubsection{Single-cell proteomics}\label{single-cell-proteomics}}

Cell sorting techniques primarily rely on extracellular proteins to
identify and separate different cell types. However, the intracellular
content of cells can still exhibit variability in experiments.
Single-cell proteomics (SCP) addresses this limitation by providing a
snapshot of the underlying processes within an individual cell,
eliminating this variability and allowing for the elucidation of the
underlying principles of cell type behavior, such as differentiation.
The immune system, for instance, relies on specialized cells for
specific functions, and the acquisition of these functions occurs
through a process called differentiation. Studying differentiation in
multiple cells can be challenging due to the inherent heterogeneity
during the initiation of differentiation. By analyzing one cell at a
time, stimulating and observing it throughout the differentiation
process, researchers can capture the entire process at a single-cell
level. However, sample preparation and computational analysis in SCP can
be tedious, and statistical testing can be challenging due to low cell
counts. In 2021, Specht et al.~developed the SCoPE2 sample preparation
and analysis pipeline, focusing on macrophage-monocyte differentiation
experiments \citep{Specht2021}. This pipeline, which reduced labor time,
served as a reference model for subsequent projects. SCP provides
insights qualitatively and quantitatively, enabling the study of drug
responses, infectious diseases, and organism development. Over the past
decade, single cell techniques became indispensable tools for global
research communities \citep{Minakshi et al., 2019}.

An early approach of qualitative analysis of the cellular proteome
involved labeling with fluorescent antibodies and imaging. The major
disadvantage of this technique was the limitation to only stain a few
proteins per cell. For quantification procedures such as single-cell
Western blots, immunoassays or CyTOF have been used. Other disadvantages
are the ability to permeate cells, accessibility and binding of the
epitope and the creation of specific antibodies for a given protein
\citep{Budnik2018}.

One of those techniques involved RNA-sequencing. Since RNA involves also
non-coding RNA, the amount of RNA is often not proportional to the
content of proteins in a cell. So the proteinaceous content of a cell
was only predicted and quantitative analysis was not possible.

\hypertarget{effectors-of-the-proteome}{%
\subsection{Effectors of the proteome}\label{effectors-of-the-proteome}}

\hypertarget{external-effectors}{%
\subsubsection{External effectors}\label{external-effectors}}

When pathogens interact with cells of an organism a cascade of
biochemical reactions takes place. These reactions influence the outcome
of the cellular proteome in regards of localization, abundance,
post-translational modifications and more (Jean Beltran et al., 2017).

\hypertarget{methods-for-quantifications}{%
\subsection{Methods for
Quantifications}\label{methods-for-quantifications}}

A holistic approach to analyze the expression can be sequencing of RNA
(RNAseq) or mass-spectrometry (MS). RNAseq refers to the task of
examining the transcriptome of bulk tissue sample with deep-sequencing
technologies. RNA in a sample includes mRNAs, non-coding RNAs, and small
RNAs. One of the main disadvantages is, that the mapping relies on the
knowledge of the DNA sequence of the observed organism (reference
genome) (Wang et al., 2009). Another disadvantage is, that the amount of
protein translated from the determined mRNA can only be estimated (Gygi
et al., 1999). However, today these predictions can be done with
computational methods such as LASSO and the accuracy of the estimate
highly depends on selection and validation of this method (Magnusson et
al., 2022).

When observing the protein ensemble of a cellular environment with a
top-down approach like DNAseq or RNAseq, the characteristics
(e.g.~post-translational modifications or charge) and local
concentration of the proteins remain hidden. The cause of this
phenomenon is that only a fraction of the DNA is transcribed into mRNA
and not all mRNA is further translated into protein. Furthermore
multiple proteins can be derived from a single DNA-sequence with
alternative splicing and different proteins coming from non related
sequences acting like building blocks can form protein complexes and/or
Protein-RNA complexes. Cells are highly efficient and do not waste
energy by producing obsolete proteins, moreover the degradation of
proteins is an important energy recovery mechanism. A bottom-up strategy
can elucidate these properties of the protein ensemble at a given time
leading to proteomics. Early methods in molecular biology were western
blots with the disadvantage of having a-priori knowledge of the analyzed
protein and quantitative protein assays like the UV-vis-, Bradford-
and/or bicinchoninic-acid-assay measuring the entire protein
concentration. When studying impact of an external signal or stressor to
a cell the abundance and properties of a protein or protein-ensemble can
enlighten the underlying pathway leading to a biological response. Since
some proteins do not even function before post-translational
modifications happen, a knowledge about these can be crucial for further
studies.

\hypertarget{mass-spectrometry}{%
\subsubsection{Mass Spectrometry}\label{mass-spectrometry}}

Mass spectrometry enables qualitative and quantitative analysis of the
entire repertoire of a biological sample. The availability of gene
sequences in databases and the ability to match proteins against those
sequences with computational methods makes it possible to identify
alterations of a sample on a protein level. These alterations can rely
on the sequence level or could be to post-translational modifications
(PTMs) such as phosphorylation, methylation or else
\citep{Aebersold2003}.

Mass spectrometers measure the mass to charge ratio (m/z) of a particle
from a fragmented larger molecule. This is achieved by a physical
procedure done with a device which is made of three major components:
the ion source, an analyzer, and a detector. The ion sources charges
molecules and accelerate them through a magnetic field. The analyzer
separates particles according to their mass to charge ratio and the
detector senses charged particles and amplifies their signal (Parker et
al., 2010). MS provides great sensitivity for traces, hence getting an
average concentration for all biomolecules in a sample. To identify
protein abundance from a single type of cell, cell sorting techniques
needed to be improved and applied to MS (Minakshi et al., 2019).
Single-cell proteomics by mass spectrometry (SCoPE-MS) uses a isobaric
labeling with a tandem mass tag (TMT) to enhance signal intensity of a
protein species. Loading of a single-cell channel (SCC) with the right
amount can be balanced to the protein concentration of a single cell,
leading to single-cell proteomics (Ye et al., 2022).

MS provides great sensitivity for traces, hence getting an average
concentration for all biomolecules in a sample. To identify protein
abundance from a single type of cell, cell sorting techniques needed to
be improved and applied to MS (Minakshi et al., 2019). Single-cell
proteomics by mass spectrometry (SCoPE-MS) uses a isobaric labeling with
a tandem mass tag (TMT) to enhance signal intensity of a protein
species. Loading of a single-cell channel (SCC) with the right amount
can be balanced to the protein concentration of a single cell, leading
to single-cell proteomics (Ye et al., 2022).

\end{document}
